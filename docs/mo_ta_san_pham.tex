% !TEX program = xelatex
\documentclass[12pt,a4paper]{article}
\usepackage{fontspec}
\setmainfont{Times New Roman}
\usepackage{geometry}
\usepackage{graphicx}
\usepackage{booktabs}
\usepackage{array}
\usepackage[table]{xcolor}
\usepackage{hyperref}
\usepackage{enumitem}
\usepackage{tikz}
\usepackage{fancyhdr}
\usepackage{titlesec}
\usepackage{amsmath}

% Page setup
\geometry{margin=2.5cm, headheight=16pt}
\pagestyle{fancy}
\fancyhf{}
\fancyhead[L]{\textcolor{gray}{Tài liệu Mô tả Sản phẩm}}
\fancyhead[R]{\textcolor{gray}{EduPath}}
\fancyfoot[C]{\thepage}

% Colors
\definecolor{primaryblue}{RGB}{30, 58, 95}
\definecolor{accentgold}{RGB}{212, 168, 83}
\definecolor{lightgray}{RGB}{245, 247, 250}

% Title formatting
\titleformat{\section}{\Large\bfseries\color{primaryblue}}{\thesection.}{0.5em}{}
\titleformat{\subsection}{\large\bfseries\color{primaryblue!80}}{\thesubsection.}{0.5em}{}

\begin{document}

% ========== TITLE PAGE ==========
\begin{titlepage}
    \centering
    \vspace*{2cm}
    
    {\Huge\bfseries\color{primaryblue} EduPath}\\[0.5cm]
    {\Large\color{accentgold} Hệ thống Tư vấn Tuyển sinh Thông minh}\\[2cm]
    
    \rule{\textwidth}{1pt}\\[0.5cm]
    {\LARGE\bfseries Tài liệu Mô tả Sản phẩm}\\[0.3cm]
    {\large Phiên bản 1.0}\\[0.5cm]
    \rule{\textwidth}{1pt}\\[3cm]
    
    \begin{tabular}{ll}
        \textbf{Dự án:} & Hỗ trợ Học sinh Định hướng Tuyển sinh \\[0.3cm]
        \textbf{Lĩnh vực:} & Phần mềm Ứng dụng \\[0.3cm]
        \textbf{Ngày tạo:} & \today \\
    \end{tabular}
    
    \vfill
    
    {\small\textcolor{gray}{Tài liệu này mô tả các tính năng của hệ thống EduPath.\\Khách hàng có thể lựa chọn các tính năng phù hợp với nhu cầu.}}
    
\end{titlepage}

% ========== TABLE OF CONTENTS ==========
\tableofcontents
\newpage

% ========== INTRODUCTION ==========
\section{Giới thiệu tổng quan}

\subsection{Bối cảnh và vấn đề}

Trong bối cảnh tuyển sinh đại học ngày càng phức tạp với nhiều phương thức xét tuyển (xét điểm thi THPT, xét học bạ, đánh giá năng lực, đánh giá tư duy...), học sinh lớp 12 gặp nhiều khó khăn trong việc:

\begin{itemize}[leftmargin=2cm]
    \item Tra cứu và so sánh điểm chuẩn giữa các trường, các ngành
    \item Đánh giá khả năng trúng tuyển của bản thân
    \item Xác định ngành học phù hợp với năng lực và sở thích
    \item Tiếp cận thông tin tư vấn một cách cá nhân hóa
\end{itemize}

\subsection{Giải pháp đề xuất}

\textbf{EduPath} là một nền tảng web tích hợp công nghệ trí tuệ nhân tạo (AI), được thiết kế để hỗ trợ học sinh trong quá trình định hướng và đưa ra quyết định tuyển sinh. Hệ thống cung cấp các công cụ tra cứu, phân tích và tư vấn dựa trên dữ liệu thực tế từ các trường đại học.

\subsection{Đối tượng sử dụng}

\begin{itemize}[leftmargin=2cm]
    \item Học sinh THPT (đặc biệt là lớp 12)
    \item Phụ huynh học sinh
    \item Giáo viên hướng nghiệp
    \item Cán bộ tư vấn tuyển sinh
\end{itemize}

\newpage

% ========== FEATURE TABLE ==========
\section{Bảng tổng hợp tính năng}

Bảng dưới đây liệt kê toàn bộ các tính năng có thể tích hợp vào hệ thống EduPath. Khách hàng có thể lựa chọn các tính năng phù hợp với mục tiêu và ngân sách của dự án.

\vspace{0.5cm}

\begingroup
\small
\renewcommand{\arraystretch}{1.3}
\begin{tabular}{|c|p{7cm}|l|l|}
\hline
\textbf{STT} & \textbf{Tính năng} & \textbf{Phân loại} & \textbf{Ghi chú} \\
\hline
\hline

% Core features
\multicolumn{4}{|l|}{\textbf{NHÓM A: TÍNH NĂNG CỐT LÕI}} \\
\hline
1 & Tra cứu điểm chuẩn đại học & Cốt lõi & Gói cơ bản \\
\hline
2 & Bài test hướng nghiệp Holland (RIASEC) & Cốt lõi & Gói cơ bản \\
\hline
3 & Trợ lý AI tư vấn tuyển sinh & Cốt lõi & Gói cơ bản \\
\hline
\hline

% Data & Analysis
\multicolumn{4}{|l|}{\textbf{NHÓM B: DỮ LIỆU \& PHÂN TÍCH}} \\
\hline
4 & So sánh điểm chuẩn nhiều trường & Mở rộng & 2-5 trường \\
\hline
5 & Biểu đồ xu hướng điểm chuẩn & Mở rộng & Theo năm \\
\hline
6 & Máy tính điểm xét tuyển & Mở rộng & Có điểm ưu tiên \\
\hline
7 & Dự đoán khả năng trúng tuyển & Nâng cao & Dùng AI/ML \\
\hline
8 & So sánh học phí các trường & Mở rộng & Theo vùng miền \\
\hline
\hline

% Personalization
\multicolumn{4}{|l|}{\textbf{NHÓM C: CÁ NHÂN HÓA}} \\
\hline
9 & Gợi ý ngành học thay thế & Nâng cao & Dựa trên điểm \\
\hline
10 & Đánh giá độ phù hợp ngành học & Nâng cao & Kết hợp Holland \\
\hline
11 & Lộ trình cá nhân hóa & Nâng cao & Roadmap riêng \\
\hline
12 & Phân tích điểm mạnh/yếu học tập & Nâng cao & Từ điểm số \\
\hline
\hline

% Utilities
\multicolumn{4}{|l|}{\textbf{NHÓM D: TIỆN ÍCH}} \\
\hline
13 & Timeline các mốc tuyển sinh & Mở rộng & Lịch biểu \\
\hline
14 & Checklist hồ sơ xét tuyển & Mở rộng & Theo trường \\
\hline
15 & Nhắc nhở deadline & Mở rộng & Thông báo \\
\hline
16 & Export báo cáo PDF & Mở rộng & Tổng hợp \\
\hline
17 & Lưu lịch sử tìm kiếm/tư vấn & Nâng cao & Cần tài khoản \\
\hline
\hline

% Advanced
\multicolumn{4}{|l|}{\textbf{NHÓM E: TÍNH NĂNG NÂNG CAO}} \\
\hline
18 & Bản đồ trường đại học & Nâng cao & Map tương tác \\
\hline
19 & Infographic nghề nghiệp & Nâng cao & Career path \\
\hline
20 & Quiz kiến thức ngành học & Nâng cao & Mini game \\
\hline

\end{tabular}
\endgroup

\newpage

% ========== DETAILED DESCRIPTIONS ==========
\section{Mô tả chi tiết các tính năng}

% ----- NHÓM A -----
\subsection{Nhóm A: Tính năng Cốt lõi}

\subsubsection{Tra cứu điểm chuẩn đại học}

\textbf{Mô tả:} Hệ thống cơ sở dữ liệu chứa thông tin điểm chuẩn từ hơn 80 trường đại học trên toàn quốc, được cập nhật theo từng năm tuyển sinh.

\textbf{Chức năng chi tiết:}
\begin{itemize}
    \item Tìm kiếm theo tên trường, mã trường, tên ngành, mã ngành
    \item Lọc theo khối thi (A00, A01, B00, C00, D01...)
    \item Lọc theo khu vực địa lý (Miền Bắc, Miền Trung, Miền Nam)
    \item Sắp xếp theo điểm chuẩn (cao $\rightarrow$ thấp hoặc ngược lại)
    \item Hiển thị thông tin chi tiết: tên trường, ngành, khối, điểm chuẩn, ghi chú
\end{itemize}

\textbf{Dữ liệu mẫu:}
\begin{center}
\begin{tabular}{|l|l|c|c|}
\hline
\textbf{Trường} & \textbf{Ngành} & \textbf{Khối} & \textbf{Điểm 2025} \\
\hline
ĐH Bách khoa HN & Khoa học Máy tính & A00 & 28.50 \\
ĐH Kinh tế Quốc dân & Kinh tế & A00 & 29.50 \\
ĐH Y Hà Nội & Y khoa & B00 & 29.00 \\
\hline
\end{tabular}
\end{center}

\vspace{0.5cm}

\subsubsection{Bài test hướng nghiệp Holland (RIASEC)}

\textbf{Mô tả:} Công cụ đánh giá xu hướng nghề nghiệp dựa trên lý thuyết Holland Code - một trong những framework được sử dụng rộng rãi nhất trong tư vấn hướng nghiệp trên thế giới.

\textbf{Cơ sở khoa học:}

Lý thuyết Holland phân loại con người và môi trường làm việc thành 6 nhóm:

\begin{center}
\begin{tabular}{|c|l|l|}
\hline
\textbf{Mã} & \textbf{Tên nhóm} & \textbf{Đặc điểm} \\
\hline
R & Realistic (Thực tế) & Thích làm việc với máy móc, công cụ \\
I & Investigative (Nghiên cứu) & Thích phân tích, tìm hiểu, khám phá \\
A & Artistic (Nghệ thuật) & Thích sáng tạo, tự do, thẩm mỹ \\
S & Social (Xã hội) & Thích giúp đỡ, giao tiếp với người khác \\
E & Enterprising (Quản lý) & Thích lãnh đạo, thuyết phục, kinh doanh \\
C & Conventional (Nghiệp vụ) & Thích công việc có tổ chức, quy trình rõ ràng \\
\hline
\end{tabular}
\end{center}

\textbf{Quy trình thực hiện:}
\begin{enumerate}
    \item Người dùng trả lời bộ câu hỏi (30-48 câu)
    \item Hệ thống tính điểm cho từng nhóm RIASEC
    \item Xác định 3 nhóm có điểm cao nhất (Holland Code)
    \item Đề xuất các ngành học và nghề nghiệp phù hợp
\end{enumerate}

\textbf{Kết quả đầu ra:}
\begin{itemize}
    \item Biểu đồ radar thể hiện điểm số 6 nhóm
    \item Mã Holland cá nhân (VD: RIA, SEC, AIS...)
    \item Danh sách ngành học được đề xuất
    \item Mô tả chi tiết từng nhóm tính cách
\end{itemize}

\vspace{0.5cm}

\subsubsection{Trợ lý AI tư vấn tuyển sinh}

\textbf{Mô tả:} Chatbot thông minh sử dụng công nghệ xử lý ngôn ngữ tự nhiên (NLP) và mô hình ngôn ngữ lớn (Large Language Model - LLM) để trả lời các câu hỏi về tuyển sinh một cách tự nhiên và cá nhân hóa.

\textbf{Công nghệ sử dụng:}
\begin{itemize}
    \item Mô hình ngôn ngữ: GPT-class model (thông qua Groq API)
    \item Xử lý ngôn ngữ tự nhiên tiếng Việt
    \item Tích hợp dữ liệu điểm chuẩn để trả lời chính xác
\end{itemize}

\textbf{Khả năng của AI:}
\begin{itemize}
    \item Trả lời câu hỏi về điểm chuẩn, ngành học, trường đại học
    \item Tư vấn lựa chọn ngành dựa trên điểm số và sở thích
    \item So sánh các trường, các ngành theo yêu cầu
    \item Giải đáp thắc mắc về quy trình xét tuyển
    \item Cung cấp thông tin về cơ hội việc làm sau tốt nghiệp
\end{itemize}

\textbf{Ví dụ hội thoại:}
\begin{quote}
\textit{Người dùng:} ``Em có 27 điểm khối A00, muốn học ngành CNTT ở TP.HCM, có những trường nào phù hợp?''

\textit{AI:} ``Với 27 điểm khối A00, em có thể xem xét các trường sau tại TP.HCM:
\begin{itemize}
    \item ĐH Khoa học Tự nhiên - ĐHQG TP.HCM: 26.5 điểm
    \item ĐH Công nghệ Thông tin - ĐHQG TP.HCM: 27.0 điểm
    \item ĐH Bách khoa TP.HCM: 27.5 điểm (cạnh tranh cao)
\end{itemize}
Em nên đăng ký 2-3 nguyện vọng để tăng cơ hội trúng tuyển...''
\end{quote}

\newpage

% ----- NHÓM B -----
\subsection{Nhóm B: Dữ liệu \& Phân tích}

\subsubsection{So sánh điểm chuẩn nhiều trường}

\textbf{Mô tả:} Công cụ cho phép người dùng đặt song song thông tin của 2-5 trường/ngành để so sánh trực quan.

\textbf{Thông tin so sánh:}
\begin{itemize}
    \item Điểm chuẩn các năm gần nhất (2023, 2024, 2025)
    \item Các khối xét tuyển
    \item Chỉ tiêu tuyển sinh
    \item Học phí (nếu có dữ liệu)
    \item Vị trí địa lý
\end{itemize}

\textbf{Giao diện:} Bảng so sánh dạng cột, dễ nhìn, có highlight điểm khác biệt.

\vspace{0.5cm}

\subsubsection{Biểu đồ xu hướng điểm chuẩn}

\textbf{Mô tả:} Trực quan hóa dữ liệu điểm chuẩn qua các năm bằng biểu đồ đường (line chart), giúp người dùng nhận biết xu hướng tăng/giảm của điểm chuẩn.

\textbf{Chức năng:}
\begin{itemize}
    \item Hiển thị điểm chuẩn 3-5 năm gần nhất
    \item So sánh nhiều ngành/trường trên cùng biểu đồ
    \item Dự đoán xu hướng năm tiếp theo (đường nét đứt)
    \item Xuất biểu đồ dưới dạng hình ảnh
\end{itemize}

\textbf{Công nghệ:} Chart.js / D3.js

\vspace{0.5cm}

\subsubsection{Máy tính điểm xét tuyển}

\textbf{Mô tả:} Công cụ tính toán điểm xét tuyển chính xác, bao gồm cả điểm ưu tiên theo khu vực và đối tượng.

\textbf{Công thức tính:}
\[
\text{Điểm xét tuyển} = \text{Điểm môn 1} + \text{Điểm môn 2} + \text{Điểm môn 3} + \text{Điểm ưu tiên}
\]

\textbf{Điểm ưu tiên khu vực:}
\begin{itemize}
    \item KV1: +0.75 điểm
    \item KV2-NT: +0.50 điểm
    \item KV2: +0.25 điểm
    \item KV3: +0.00 điểm
\end{itemize}

\textbf{Điểm ưu tiên đối tượng:}
\begin{itemize}
    \item Nhóm ưu tiên 1 (UT1): +2.0 điểm
    \item Nhóm ưu tiên 2 (UT2): +1.0 điểm
\end{itemize}

\vspace{0.5cm}

\subsubsection{Dự đoán khả năng trúng tuyển}

\textbf{Mô tả:} Sử dụng thuật toán Machine Learning để ước tính xác suất trúng tuyển của học sinh vào một ngành/trường cụ thể.

\textbf{Dữ liệu đầu vào:}
\begin{itemize}
    \item Điểm dự kiến của học sinh (hoặc điểm thi thật)
    \item Ngành và trường muốn xét
    \item Dữ liệu điểm chuẩn lịch sử 5 năm
    \item Xu hướng biến động điểm
\end{itemize}

\textbf{Kết quả đầu ra:}
\begin{itemize}
    \item Phần trăm khả năng trúng tuyển (VD: 75\%)
    \item Đánh giá mức độ: An toàn / Vừa sức / Có rủi ro / Khó đậu
    \item Gợi ý ngành thay thế nếu khả năng thấp
\end{itemize}

\textbf{Lưu ý:} Kết quả chỉ mang tính tham khảo, không đảm bảo chính xác 100\%.

\vspace{0.5cm}

\subsubsection{So sánh học phí các trường}

\textbf{Mô tả:} Cung cấp thông tin học phí của các trường đại học, giúp học sinh và phụ huynh cân nhắc yếu tố tài chính.

\textbf{Thông tin bao gồm:}
\begin{itemize}
    \item Học phí theo năm / theo tín chỉ
    \item Phân biệt chương trình đại trà và chất lượng cao
    \item Chi phí sinh hoạt ước tính theo vùng miền
    \item Thông tin học bổng (nếu có)
\end{itemize}

\newpage

% ----- NHÓM C -----
\subsection{Nhóm C: Cá nhân hóa}

\subsubsection{Gợi ý ngành học thay thế}

\textbf{Mô tả:} Khi điểm của học sinh không đủ để vào ngành mong muốn, hệ thống sẽ tự động gợi ý các ngành có điểm chuẩn thấp hơn nhưng có liên quan về lĩnh vực.

\textbf{Ví dụ:}
\begin{quote}
``Điểm của bạn (26.5) chưa đủ để vào ngành Y khoa (29.0). Bạn có thể xem xét các ngành thay thế:
\begin{itemize}
    \item Điều dưỡng: 24.5 điểm
    \item Y tế công cộng: 25.0 điểm
    \item Dược học: 27.0 điểm
\end{itemize}''
\end{quote}

\vspace{0.5cm}

\subsubsection{Đánh giá độ phù hợp ngành học}

\textbf{Mô tả:} Kết hợp kết quả bài test Holland với điểm số học tập để đưa ra đánh giá mức độ phù hợp của học sinh với từng ngành học.

\textbf{Tiêu chí đánh giá:}
\begin{itemize}
    \item Sự phù hợp về tính cách (từ Holland test): 40\%
    \item Năng lực học tập (từ điểm số): 40\%
    \item Xu hướng thị trường việc làm: 20\%
\end{itemize}

\textbf{Kết quả:} Điểm phù hợp từ 0-100\% cho mỗi ngành.

\vspace{0.5cm}

\subsubsection{Lộ trình cá nhân hóa}

\textbf{Mô tả:} Sau khi hoàn thành các bài đánh giá, hệ thống tạo ra một ``roadmap'' riêng cho từng học sinh, bao gồm:

\begin{itemize}
    \item Mục tiêu điểm cần đạt
    \item Danh sách ngành/trường phù hợp (sắp xếp theo độ ưu tiên)
    \item Các bước chuẩn bị hồ sơ
    \item Timeline các mốc quan trọng
    \item Tài nguyên học tập gợi ý
\end{itemize}

\vspace{0.5cm}

\subsubsection{Phân tích điểm mạnh/yếu học tập}

\textbf{Mô tả:} Người dùng nhập điểm các môn học (có thể là điểm trung bình 3 năm THPT), hệ thống phân tích và đưa ra nhận xét.

\textbf{Phân tích bao gồm:}
\begin{itemize}
    \item Môn học có thế mạnh / cần cải thiện
    \item Xu hướng điểm số qua các năm (tiến bộ / ổn định / giảm sút)
    \item Gợi ý khối thi phù hợp với năng lực
    \item Đề xuất môn cần tập trung ôn luyện
\end{itemize}

\newpage

% ----- NHÓM D -----
\subsection{Nhóm D: Tiện ích}

\subsubsection{Timeline các mốc tuyển sinh}

\textbf{Mô tả:} Lịch biểu trực quan hiển thị các mốc thời gian quan trọng trong năm tuyển sinh.

\textbf{Các mốc bao gồm:}
\begin{itemize}
    \item Thời gian đăng ký thi THPT
    \item Ngày thi THPT Quốc gia
    \item Thời gian nộp hồ sơ xét tuyển sớm
    \item Ngày công bố điểm thi
    \item Thời gian đăng ký nguyện vọng
    \item Ngày công bố kết quả xét tuyển
    \item Thời hạn nhập học
\end{itemize}

\vspace{0.5cm}

\subsubsection{Checklist hồ sơ xét tuyển}

\textbf{Mô tả:} Danh sách các giấy tờ, thủ tục cần chuẩn bị cho từng phương thức xét tuyển và từng trường.

\textbf{Phân loại theo phương thức:}
\begin{itemize}
    \item Xét điểm thi THPT
    \item Xét học bạ
    \item Xét đánh giá năng lực (ĐHQG)
    \item Xét tuyển thẳng / ưu tiên xét tuyển
\end{itemize}

\vspace{0.5cm}

\subsubsection{Nhắc nhở deadline}

\textbf{Mô tả:} Hệ thống gửi thông báo (email hoặc notification trên web) nhắc nhở người dùng về các mốc thời gian sắp đến.

\textbf{Yêu cầu:} Người dùng cần đăng ký tài khoản và cung cấp email.

\vspace{0.5cm}

\subsubsection{Export báo cáo PDF}

\textbf{Mô tả:} Cho phép người dùng xuất toàn bộ kết quả (bài test Holland, danh sách trường gợi ý, lộ trình...) ra file PDF để lưu trữ hoặc in ấn.

\textbf{Nội dung báo cáo:}
\begin{itemize}
    \item Thông tin cá nhân
    \item Kết quả bài test Holland (biểu đồ + mô tả)
    \item Danh sách ngành/trường đề xuất
    \item Lộ trình cá nhân hóa
    \item Ghi chú và khuyến nghị
\end{itemize}

\vspace{0.5cm}

\subsubsection{Lưu lịch sử tìm kiếm/tư vấn}

\textbf{Mô tả:} Người dùng có tài khoản có thể lưu lại các kết quả tìm kiếm, cuộc hội thoại với AI để xem lại sau.

\textbf{Yêu cầu:} Cần xây dựng hệ thống đăng nhập và cơ sở dữ liệu người dùng.

\newpage

% ----- NHÓM E -----
\subsection{Nhóm E: Tính năng Nâng cao}

\subsubsection{Bản đồ trường đại học}

\textbf{Mô tả:} Bản đồ tương tác (interactive map) hiển thị vị trí các trường đại học trên toàn quốc.

\textbf{Chức năng:}
\begin{itemize}
    \item Click vào marker để xem thông tin trường
    \item Lọc theo khu vực, loại trường (công lập / dân lập)
    \item Hiển thị khoảng cách từ vị trí người dùng
    \item Tích hợp Google Maps / Leaflet
\end{itemize}

\vspace{0.5cm}

\subsubsection{Infographic nghề nghiệp}

\textbf{Mô tả:} Mỗi ngành học có một trang infographic riêng, trình bày trực quan về:

\begin{itemize}
    \item Mô tả ngành học
    \item Các vị trí công việc có thể đảm nhận
    \item Mức lương trung bình (theo kinh nghiệm)
    \item Lộ trình thăng tiến (career path)
    \item Kỹ năng cần có
    \item Các trường đào tạo ngành này
\end{itemize}

\vspace{0.5cm}

\subsubsection{Quiz kiến thức ngành học}

\textbf{Mô tả:} Mini quiz giúp học sinh tìm hiểu sâu hơn về ngành học mình quan tâm trước khi quyết định.

\textbf{Nội dung quiz:}
\begin{itemize}
    \item Kiến thức cơ bản về ngành
    \item Các môn học sẽ được đào tạo
    \item Yêu cầu năng lực và tố chất
    \item Thực tế công việc sau tốt nghiệp
\end{itemize}

\textbf{Mục đích:} Giúp học sinh hiểu rõ ngành học, tránh chọn sai do thiếu thông tin.

\newpage

% ========== ARCHITECTURE ==========
\section{Kiến trúc hệ thống}

\subsection{Sơ đồ tổng quan}

\begin{center}
\begin{tikzpicture}[
    box/.style={rectangle, draw=primaryblue, fill=lightgray, minimum width=3cm, minimum height=1cm, align=center, rounded corners=3pt},
    arrow/.style={->, thick, primaryblue}
]
    % Frontend
    \node[box] (frontend) at (0, 0) {Frontend\\(HTML/CSS/JS)};
    
    % Backend
    \node[box] (backend) at (5, 0) {Backend Logic\\(JavaScript)};
    
    % AI
    \node[box] (ai) at (10, 0) {AI Engine\\(Groq LLM API)};
    
    % Database
    \node[box] (db) at (5, -2.5) {Database\\(JSON/CSV)};
    
    % User
    \node[box, fill=accentgold!30] (user) at (-4, 0) {Người dùng};
    
    % Arrows
    \draw[arrow] (user) -- (frontend);
    \draw[arrow, <->] (frontend) -- (backend);
    \draw[arrow, <->] (backend) -- (ai);
    \draw[arrow, <->] (backend) -- (db);
    
\end{tikzpicture}
\end{center}

\subsection{Công nghệ sử dụng}

\begin{center}
\begin{tabular}{|l|l|l|}
\hline
\rowcolor{primaryblue}
\textcolor{white}{\textbf{Thành phần}} & \textcolor{white}{\textbf{Công nghệ}} & \textcolor{white}{\textbf{Mô tả}} \\
\hline
Giao diện & HTML5, CSS3, JavaScript & Responsive, tương thích đa thiết bị \\
\hline
Trực quan hóa & Chart.js & Vẽ biểu đồ xu hướng, radar \\
\hline
Bản đồ & Leaflet / Google Maps & Hiển thị vị trí trường \\
\hline
AI/NLP & Groq API (LLM) & Chatbot thông minh \\
\hline
Dữ liệu & JSON, CSV & Lưu trữ điểm chuẩn \\
\hline
PDF Export & jsPDF & Xuất báo cáo \\
\hline
\end{tabular}
\end{center}

\newpage

% ========== CONCLUSION ==========
\section{Kết luận}

Tài liệu này đã trình bày tổng quan về hệ thống \textbf{EduPath} - Nền tảng Tư vấn Tuyển sinh Thông minh, bao gồm:

\begin{itemize}
    \item 3 tính năng cốt lõi (tra cứu điểm, test Holland, chatbot AI)
    \item 17 tính năng mở rộng và nâng cao
    \item Kiến trúc kỹ thuật của hệ thống
\end{itemize}

Khách hàng có thể lựa chọn các tính năng phù hợp với mục tiêu dự án và ngân sách. Đội ngũ phát triển sẵn sàng tư vấn và hỗ trợ trong suốt quá trình triển khai.

\vspace{1cm}

\begin{center}
\rule{0.5\textwidth}{0.5pt}

\textit{Cảm ơn quý khách đã quan tâm đến sản phẩm EduPath.}
\end{center}

\end{document}
